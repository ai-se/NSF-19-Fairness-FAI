\label{key}The University of Alabama's Department of Computer Science has access to a wealth of computing resources at the university level, the college level, and the departmental level.  At the institutional level students and faculty have access to an enterprise Sun Solaris system for academic applications.  It also maintains a high performance computing cluster of 130 dual-processor blade servers.  All networking on campus is switched 100-gig Ethernet with fiber connections between buildings.  

These resources are supplemented at the college level by general computing labs (PC and Unix) that are available for all students. The Department also maintains its own computing laboratories with hardware and software that is specific to the needs of the discipline. There are no issues with respect to computing facilities or resources for any of the research programs within the department.

The Software Engineering research group has two laboratories. These newly renovated laboratories contain over 3,000 ft\textsuperscript{2} of research space and include 20 cubicles, a meeting room with projector and multiple printing facilities (including a recently acquired large plotter). In addition to a number of workstations for individual graduate students, the lab houses a Linux server with one terabyte of RAID-5 storage (with tape backup).

The Aging Infrastructure Systems Center of Excellence occupies 7,500 ft\textsuperscript{2} of research space to facilitate four laboratories, including the Information Technology Innovation Lab.  The Information Technology Innovation Lab has over 100 workstations for research assistants, five A/V equipped breakout team rooms, and a conference room.  Video and teleconferencing is available in all the breakout and conference rooms.   The center has a cluster of servers that virtualizes individual project servers along with code repositories and document sharing.  The staff of the center provides technical assistance for both public domain and proprietary software development platforms. 

At the university level, the Office of Information Technology (oit) manages the computing and storage infrastructure for the university. As part of this operation, they support low-cost, reliable processing and data storage nodes and low-costs to university researchers. This computing infrastructure has a redundant, off-site backup to ensure continuity of operation. We anticipate purchasing capacity from OIT to support the proposed infrastructure.
