The Systematic Literature Review (SLR) process is a means for analyzing published evidence to draw conclusions about a phenomenon of interest. 
The prevalence of empirical research in software engineering (SE) makes it well-suited for SLRs. 
While the popularity of SLRs has been increasing within the SE community in recent years, there are still significant barriers to more widespread adoption. 
We propose to build the Systematic Literature Review Authoring Infrastructure and Toolkit for Software Engineering (SAInT) to address these barriers. 
(1) SAInT will provide support to automate the SLR process through back-end data storage, interchange and persistence as well as through front-end user tools. 
SE researchers will have the tools for searching multiple databases to identify, extract and compile information from existing research. 
Likewise, automation will allow researchers to discover, backtrack and iterate among the SLR phases is needed. 
(2) SAInT will reduce the effects of single-researcher bias in the SLR process, by providing support for collaborative SLRs.
There is a need to enable collaboration both among co-located teams (e.g. PhD students and their advisor) and among distributed teams (e.g. researchers from different countries). 
(3) It is likely that an article may be relevant to multiple SLRs. 
Because there is no central repository to store extracted data, researchers must repeat the data extraction for each SLR. 
SAInT will provide persistent storage of extracted data to reduce effort, by enabling a researcher to extract only the additional data relevant to the new research question(s), and to facilitate collaboration by allowing researchers to identify others working on similar topics. 
In addition, central storage of data will facilitate the goal of making SLRs into living documents that can evolve as new research is published. 
Researchers can more easily integrate new findings with the existing results by taking advantage of the access to stored data.


The goal of this CI-NEW proposal is to build upon our CI-P results to implement SAInT, an SLR support infrastructure for the SE domain, that removes barriers and encourages more widespread adoption of the SLR approach. 
SAInT will reduce the resource impedance that exists today as SE researchers rely on a labor-intensive process to ensure the rigor required to conduct systematic, comprehensive, and reproducible literature reviews. 
SE researchers will benefit from SAInT because its tools enable free-standing, independent reviews that summarize and integrate existing evidence, identify gaps in current research and provide a framework for positioning new research. 
Moreover SAInT will provide a common interchange of SLR metadata across toolsmiths and a community around which a SE SLR ecosystem can advance.

Keywords: Systematic Literature Reviews, Software Engineering, Web-based Infrastructure

\textbf{Intellectual Merit :}
High-quality literature reviews, whether systematic or ad hoc are an integral part of the research process. 
By enabling the execution of SLRs, SAInT will help to ensure that researchers identify a complete and unbiased set of literature. 
Moreover, SAInT will serve as a central repository that is easily updated, fosters additional research, and facilitates collaboration among geographically distributed researchers.

\textbf{Broader Impacts :}
This project will lower the barrier to performing SLRs. 
By removing the barriers currently faced by SE researchers, SAInT will allow a larger portion of the SE community to participate in the conduct of SLRs. 
SAInT will be especially helpful to PhD students who conduct a literature review as part of their thesis development. 
Furthermore, SAInT will increase the prevalence of summarized results that can inform research and practice. 
SAInT will also make those results more readily available to the wider audience. 
Finally, other research domains, such as healthcare, experience many of the same barriers as SE. 
Thus, SAInT potentially can address needs in other SLR communities.