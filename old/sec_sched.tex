The schedule in Table~\ref{table-schedule} provides an overview of the number of development sprints planned for each quarter.  
It also provide an overview of other major activities planned each quarter such as Program Coordination and reviews by external (i.e., Non-UA) SLR  tool developers.  
Other reviews will be conducted at least every 9 months by external SLR authors and practitioners. 
The workshops for External Tool Developers and SLR Authors are also shown.   
The last item on the schedule shows the efforts that will take place to ensure the infrastructure is sustainable beyond the life of this grant. 

The project schedule considers multiple constraints and goals, for example:
\vspace{-8pt}
\begin{itemize}
	\item SAInT Feature Integration will ramp up during the first quarter (conducting 3 sprints) and then reach a steady state of 4 sprints for each of the next four quarters.  
	At the mid-point of the grant period, we will conduct a workshop with non-UA tool builders and SLR authors.  
	\item Preparing for the workshop and responding to feedback will reduce the number of sprints in the Winter 2017-18 quarter (after which 4 sprints per quarter will resume).  
	\item In the last 3 quarters of the project, the number of sprints will be ramped down as more user testing and training activities ramp up.
	\vspace{-4pt}
\end{itemize}
\vspace{-4pt}

\begin{table}
	\centering
	\caption{Project Schedule}
	\label{table-schedule}
	\begin{table}[!bth]
\caption{Project Schedule}
\label{tab:Schedule}
{\footnotesize 
\begin{tabular}{p{.48\textwidth}|c@{~}c@{~}c@{~}c|c@{~}c@{~}c@{~}c|c@{~}c@{~}c@{~}c|}
\cline{2-13}
	&   \multicolumn{12}{c|}{\textbf{Quarters}}  \\
	&   \multicolumn{4}{c}{Year 1}      & \multicolumn{4}{c}{Year 2}     & \multicolumn{4}{c|}{Year 3}      \\
	Tasks                          & Fall & Wint & Spr & Sum & Fall & Wint & Spr & Sum & Fall & Wint & Spr & Sum \\
\hline
	Program Coordination           & x    & x    & x   & x   & x    & x    & x   & x   & x    & x    & x   & x   \\
\hline
	\textbf{Initialize} - Analyze \& choose AI/Text-mining tools \& Identify data outputs for SLR phases       & x    & x    & x   & x   &     &     &    &    &     &     &    &    \\
\hline
	\textbf{Compose} - AI/Text-mining tools \& Data model for SLR phases       & x    & x    & x   & x   & x    &  x   & x   & x   & x    &  x   &x    & x  \\
\hline
	\textbf{Popularize} - Conduct one-on-one case studies  &     &     &    & x   &     &  x   &    & x   &     & x    &    & x   \\
\hline
	\textbf{Popularize} - Workshops                     &      &      &      &  x   &      &     &     &     &      &      &     &   x  \\
\hline
	\textbf{Support} - Use Helpdesk &      &      &      &     & x     & x    &x     & x    & x     & x     & x    & x    \\
\hline
	\textbf{Audit} - Tune AI/Text-mining approaches \& Conduct human-based studies                     &  x    &   x   &   x  &  x   &    x  &   x  &  x  &   x  &   x   &  x   &   x  & x   \\\cline{2-13}

\end{tabular}
}
\end{table}
\end{table}